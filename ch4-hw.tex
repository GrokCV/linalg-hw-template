% !TEX program = xelatex
% ===================================================================
% 《线性代数》课程作业模板
% ===================================================================
%
%  【学生使用说明】
%  1. 在下方的【学生信息区】将 "请在此处填写..." 替换为你的真实信息。
%  2. 注意:以下所有题目均为模板占位符,请自己输入实际的题目,并作答.
%
% ===================================================================


\documentclass[a4paper, 12pt]{ctexart}

% --- 课程信息 (通常无需修改) ---
\newcommand{\CourseInfo}{线性代数} % 用于标题行的大标题
\newcommand{\CourseName}{线性代数·2025秋·第 4 章}               % 用于页眉左边
\newcommand{\InstructorName}{戴一冕}
\newcommand{\AssignmentTitle}{第 4 章 \quad 线性空间}

% ===================================================================
%  第一步:请在这里填写你的学生信息
% ===================================================================
\newcommand{\StudentName}{xxx}
\newcommand{\StudentID}{xxxxxxx}
\newcommand{\AssignmentDate}{2025年xx月xx日} % 默认是提交日期

% --- 导入所有配置 (请勿修改) ---
% ===================================================================
%  config.tex -- The All-in-One Style & Info Configuration File
% ===================================================================

% --- PART 1: CORE PACKAGES ---
\usepackage[left=2.5cm, right=2.5cm, top=2.5cm, bottom=3cm]{geometry}
\usepackage{fancyhdr}
\usepackage{lastpage}
\usepackage{hyperref}
\usepackage[table]{xcolor}
\usepackage{amsmath, amssymb}
\usepackage{tikz}
\usetikzlibrary{shadows}
\usepackage[shortlabels]{enumitem}
\usepackage[many]{tcolorbox}
\usepackage{pifont, bbding}


% --- PART 2: USER INFORMATION AREA (WITH DEFAULTS) ---
\providecommand{\CourseInfo}{(请在main.tex中填写课程信息)}
\providecommand{\CourseName}{(课程名)}
\providecommand{\InstructorName}{(教师名)}
\providecommand{\AssignmentTitle}{(作业标题)}
\providecommand{\StudentName}{(请在main.tex中填写姓名)}
\providecommand{\StudentID}{(请在main.tex中填写学号)}
\providecommand{\StudentClass}{(请在main.tex中填写班级)}
\providecommand{\AssignmentDate}{\today}




% --- PART 3: HEADER, FOOTER, AND HYPERLINK SETUP ---
\setlength{\headheight}{15pt}
\pagestyle{fancy}
\fancyhf{}
\fancyhead[L]{\CourseName}
\fancyhead[R]{主讲: \InstructorName}
\fancyfoot[L]{课程主页:\href{https://grokcv.ai/teaching/}{https://grokcv.ai/teaching/}}
\fancyfoot[C]{}
\fancyfoot[R]{第 \thepage\ 页 / 共 \pageref{LastPage} 页}
\renewcommand{\headrulewidth}{0.4pt}
\renewcommand{\footrulewidth}{0.4pt}
\hypersetup{colorlinks, linkcolor=winered, citecolor=winered, urlcolor=winered}


% --- PART 4: AUTOMATIC TITLE BLOCK COMMAND ---
\newcommand{\makeHomeworkTitle}{%
    \vspace*{0.2cm}
    \begin{center}
        {\huge \songti \bfseries \CourseInfo} \\ \vspace{1.5em} 
        {\Large \songti \bfseries \AssignmentTitle} \\ \vspace{2.5em}
    \end{center}
    \noindent
    \begin{tabular*}{\textwidth}{@{\extracolsep{\fill}} ll}
        \songti 姓名: \StudentName & \songti 班级: \StudentClass \\ 
        \songti 学号: \StudentID & \songti 日期: \AssignmentDate \\ 
    \end{tabular*}
    \vspace{2em}\hrule\vspace{2em}
}


% --- PART 5: VISUAL STYLE SETUP (Colors, Theorems, Lists) ---
% 核心颜色定义
\definecolor{structurecolor}{RGB}{51,102,153}
\definecolor{main}{RGB}{0,102,51}
\definecolor{second}{RGB}{204,102,0}
\definecolor{third}{RGB}{0,128,128}
\definecolor{winered}{rgb}{0.5,0,0}

% 定理、定义环境
\renewcommand{\proofname}{证明}
\newcommand{\definitionname}{定义}
\newcommand{\theoremname}{定理}
\newcommand{\exercisename}{练习}
\newcommand{\hintname}{提示 \ding{182}}

% 1统一使用宋体 (fontupper={\songti ...})
\tcbset{
    common/.style={fontupper={\songti\fontseries{m}\selectfont}, coltitle=white, colback=gray!5, boxrule=0.5pt, fonttitle=\songti, enhanced, breakable, top=8pt, before skip=10pt, after skip=10pt, attach boxed title to top left={yshift=-0.1in, xshift=0.15in}, boxed title style={boxrule=0pt, colframe=white, arc=0pt, outer arc=0pt}, separator sign={.}},
    defstyle/.style={common, colframe=main, colback=main!5, colbacktitle=main},
    thmstyle/.style={common, colframe=second, colback=second!5, colbacktitle=second},
    hintstyle/.style={common, colframe=third, colback=third!5, colbacktitle=third}
}
\DeclareTColorBox[auto counter, number within=section]{definition}{ o m }{defstyle, IfValueTF={#1}{title={\definitionname~\thetcbcounter\ (#1)}}{title={\definitionname~\thetcbcounter}}, IfValueT={#2}{label={#2}}}
\DeclareTColorBox[auto counter, number within=section]{theorem}{ o m }{thmstyle, IfValueTF={#1}{title={\theoremname~\thetcbcounter\ (#1)}}{title={\theoremname~\thetcbcounter}}, IfValueT={#2}{label={#2}}}

\newenvironment{hint}[1][]{%
    \def\temphintarg{#1}%
    \def\tempempty{}%
    \ifx\temphintarg\tempempty
        \begin{tcolorbox}[hintstyle, title={\hintname}]
    \else
        \begin{tcolorbox}[hintstyle, title={\hintname~(#1)}]
    \fi
}{%
    \end{tcolorbox}
}


% 练习、证明环境
\newcounter{exercise}[section]
\renewcommand{\theexercise}{\thesection.\arabic{exercise}}
\newenvironment{exercise}[1][]{%
    \par\refstepcounter{exercise}%
    \noindent{\songti\color{main}\exercisename\ \theexercise #1}\ \rmfamily %
}{%
    \par\ignorespacesafterend
}

% 2. [新增] 作答范例专用环境 (不带编号)
\newenvironment{exampleexercise}[1][]{%
    \par\noindent{\songti\color{main}\textbf{#1}}\ \rmfamily %
}{%
    \par\ignorespacesafterend
}


% --- End of config.tex ---

% ===================================================================
%  正文开始
% ===================================================================
\begin{document}

% --- 生成标题区 ---
\makeHomeworkTitle

% --- 说明框 ---
% \begin{tcolorbox}[
%     hintstyle,
%     colframe=winered, 
%     colback=winered!5!white, 
%     colbacktitle=winered, 
%     fontupper=\bfseries, 
%     title={\textbf{重要说明}}
% ]
%     以下所有题目均为模板占位符,请自己输入实际的题目,并作答
% \end{tcolorbox}

% ===================================================================
%  作业题目区
% ===================================================================
\section*{第2题}
\begin{exercise}
2. 在 $\mathbb{R}^n$ 中,分别满足下列条件的向量 $(x_1, x_2, \dots, x_n)$ 的集合能否构成 $\mathbb{R}^n$ 的子空间? \\
$(1) \quad x_1 + x_2 + \dots + x_n = 0$; \\
$(2) \quad x_1 + x_2 + \dots + x_n = 1$.
\end{exercise}

\noindent\textbf{解:}







\section*{第4题}
\begin{exercise}
4. 在线性空间 $C[a,b]$ 中,考察下列向量组是否线性相关,并求它们的秩。 \\
$(1) \quad \cos^2 x, \sin^2 x$; \\
$(2) \quad \cos^2 x, \cos 2x, 1$.


\end{exercise}
\noindent\textbf{解:}









\section*{第6题}
\begin{exercise}
6. 试证由 $\mathbb{R}^3$ 中向量 $\alpha_1 = (0,1,1)$, $\alpha_2 = (1,0,1)$, $\alpha_3 = (1,1,0)$ 所生成的线性空间就是 $\mathbb{R}^3$ 本身.

\end{exercise}
\noindent\textbf{解:}













\section*{第7题}
\begin{exercise}
7. 设在 $\mathbb{R}^4$ 中由向量组 $\alpha_1 = (1,1,0,0)$, $\alpha_2 = (1,0,1,1)$ 所生成的子空间记为 $V_1$。由向量组 $\beta_1 = (2,-1,3,3)$, $\beta_2 = (0,1,-1,-1)$ 所生成的子空间记为 $V_2$,试证 $V_1 = V_2$.


\end{exercise}
\noindent\textbf{解:}














\section*{第9题}
\begin{exercise}
9. 证明 $\varepsilon_1, \varepsilon_2, \varepsilon_3, \varepsilon_4$ 组成 $\mathbb{R}^4$ 的一个基,并求 $\beta$ 在这个基下的坐标: \\
$(1) \quad  \varepsilon_1 = (1,1,1,1), \varepsilon_2 = (1,1,-1,-1), \varepsilon_3 = (1,-1,1,-1), \varepsilon_4 = (1,-1,-1,1), \beta = (1,2,1,1).$

\end{exercise}
\noindent\textbf{解:}














\section*{第11题}
\begin{exercise}
11. 在 $\mathbb{R}^4$ 中,求基 $[\varepsilon_1, \varepsilon_2, \varepsilon_3, \varepsilon_4]$ 到基 $[\eta_1, \eta_2, \eta_3, \eta_4]$ 的过渡矩阵,并求 $\xi$ 在指定基下的坐标: \\
\[
\text{(1) } 
\left\{
\begin{array}{l}
\varepsilon_1 = (1,0,0,0), \quad \eta_1 = (2,1,-1,1), \\
\varepsilon_2 = (0,1,0,0), \quad \eta_2 = (0,3,1,0), \\
\varepsilon_3 = (0,0,1,0), \quad \eta_3 = (5,3,2,1), \\
\varepsilon_4 = (0,0,0,1), \quad \eta_4 = (6,6,1,3),
\end{array}
\right.
\]
求向量 $\xi = (x_1, x_2, x_3, x_4)$ 在基 $[\eta_1, \eta_2, \eta_3, \eta_4]$ 下的坐标。

\end{exercise}
\noindent\textbf{解:}
\








\end{document}
% ===================================================================
%  文档结束
% ===================================================================