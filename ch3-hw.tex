% !TEX program = xelatex
% ===================================================================
% 《线性代数》课程作业模板
% ===================================================================
%
%  【学生使用说明】
%  1. 在下方的【学生信息区】将 "请在此处填写..." 替换为你的真实信息。
%  2. 注意:以下所有题目均为模板占位符,请自己输入实际的题目,并作答.
%
% ===================================================================


\documentclass[a4paper, 12pt]{ctexart}

\usepackage{mathdots}
\usepackage{mathtools} % 加载 mathtools 会自动加载 amsmath

% --- 课程信息 (通常无需修改) ---
\newcommand{\CourseInfo}{线性代数} % 用于标题行的大标题
\newcommand{\CourseName}{线性代数·2025秋·第 3 章}               % 用于页眉左边
\newcommand{\InstructorName}{戴一冕}
\newcommand{\AssignmentTitle}{第 3 章 \quad 线性方程组}

% ===================================================================
%  第一步:请在这里填写你的学生信息
% ===================================================================
\newcommand{\StudentName}{xxx}
\newcommand{\StudentID}{xxxxxxx}
\newcommand{\AssignmentDate}{2025年xx月xx日} % 默认是提交日期

% --- 导入所有配置 (请勿修改) ---
% ===================================================================
%  config.tex -- The All-in-One Style & Info Configuration File
% ===================================================================

% --- PART 1: CORE PACKAGES ---
\usepackage[left=2.5cm, right=2.5cm, top=2.5cm, bottom=3cm]{geometry}
\usepackage{fancyhdr}
\usepackage{lastpage}
\usepackage{hyperref}
\usepackage[table]{xcolor}
\usepackage{amsmath, amssymb}
\usepackage{tikz}
\usetikzlibrary{shadows}
\usepackage[shortlabels]{enumitem}
\usepackage[many]{tcolorbox}
\usepackage{pifont, bbding}


% --- PART 2: USER INFORMATION AREA (WITH DEFAULTS) ---
\providecommand{\CourseInfo}{(请在main.tex中填写课程信息)}
\providecommand{\CourseName}{(课程名)}
\providecommand{\InstructorName}{(教师名)}
\providecommand{\AssignmentTitle}{(作业标题)}
\providecommand{\StudentName}{(请在main.tex中填写姓名)}
\providecommand{\StudentID}{(请在main.tex中填写学号)}
\providecommand{\StudentClass}{(请在main.tex中填写班级)}
\providecommand{\AssignmentDate}{\today}




% --- PART 3: HEADER, FOOTER, AND HYPERLINK SETUP ---
\setlength{\headheight}{15pt}
\pagestyle{fancy}
\fancyhf{}
\fancyhead[L]{\CourseName}
\fancyhead[R]{主讲: \InstructorName}
\fancyfoot[L]{课程主页:\href{https://grokcv.ai/teaching/}{https://grokcv.ai/teaching/}}
\fancyfoot[C]{}
\fancyfoot[R]{第 \thepage\ 页 / 共 \pageref{LastPage} 页}
\renewcommand{\headrulewidth}{0.4pt}
\renewcommand{\footrulewidth}{0.4pt}
\hypersetup{colorlinks, linkcolor=winered, citecolor=winered, urlcolor=winered}


% --- PART 4: AUTOMATIC TITLE BLOCK COMMAND ---
\newcommand{\makeHomeworkTitle}{%
    \vspace*{0.2cm}
    \begin{center}
        {\huge \songti \bfseries \CourseInfo} \\ \vspace{1.5em} 
        {\Large \songti \bfseries \AssignmentTitle} \\ \vspace{2.5em}
    \end{center}
    \noindent
    \begin{tabular*}{\textwidth}{@{\extracolsep{\fill}} ll}
        \songti 姓名: \StudentName & \songti 班级: \StudentClass \\ 
        \songti 学号: \StudentID & \songti 日期: \AssignmentDate \\ 
    \end{tabular*}
    \vspace{2em}\hrule\vspace{2em}
}


% --- PART 5: VISUAL STYLE SETUP (Colors, Theorems, Lists) ---
% 核心颜色定义
\definecolor{structurecolor}{RGB}{51,102,153}
\definecolor{main}{RGB}{0,102,51}
\definecolor{second}{RGB}{204,102,0}
\definecolor{third}{RGB}{0,128,128}
\definecolor{winered}{rgb}{0.5,0,0}

% 定理、定义环境
\renewcommand{\proofname}{证明}
\newcommand{\definitionname}{定义}
\newcommand{\theoremname}{定理}
\newcommand{\exercisename}{练习}
\newcommand{\hintname}{提示 \ding{182}}

% 1统一使用宋体 (fontupper={\songti ...})
\tcbset{
    common/.style={fontupper={\songti\fontseries{m}\selectfont}, coltitle=white, colback=gray!5, boxrule=0.5pt, fonttitle=\songti, enhanced, breakable, top=8pt, before skip=10pt, after skip=10pt, attach boxed title to top left={yshift=-0.1in, xshift=0.15in}, boxed title style={boxrule=0pt, colframe=white, arc=0pt, outer arc=0pt}, separator sign={.}},
    defstyle/.style={common, colframe=main, colback=main!5, colbacktitle=main},
    thmstyle/.style={common, colframe=second, colback=second!5, colbacktitle=second},
    hintstyle/.style={common, colframe=third, colback=third!5, colbacktitle=third}
}
\DeclareTColorBox[auto counter, number within=section]{definition}{ o m }{defstyle, IfValueTF={#1}{title={\definitionname~\thetcbcounter\ (#1)}}{title={\definitionname~\thetcbcounter}}, IfValueT={#2}{label={#2}}}
\DeclareTColorBox[auto counter, number within=section]{theorem}{ o m }{thmstyle, IfValueTF={#1}{title={\theoremname~\thetcbcounter\ (#1)}}{title={\theoremname~\thetcbcounter}}, IfValueT={#2}{label={#2}}}

\newenvironment{hint}[1][]{%
    \def\temphintarg{#1}%
    \def\tempempty{}%
    \ifx\temphintarg\tempempty
        \begin{tcolorbox}[hintstyle, title={\hintname}]
    \else
        \begin{tcolorbox}[hintstyle, title={\hintname~(#1)}]
    \fi
}{%
    \end{tcolorbox}
}


% 练习、证明环境
\newcounter{exercise}[subsection]
\renewcommand{\theexercise}{\thesubsection.\arabic{exercise}}
\newenvironment{proof}{\par\noindent\textbf{\color{second}\proofname:}\ \itshape}{\par}
\newenvironment{exercise}[1][]{%
    \par\refstepcounter{exercise}%
    \noindent{\songti\color{main}\exercisename\ \theexercise #1}\ \rmfamily %
}{%
    \par\ignorespacesafterend
}

% 2. [新增] 作答范例专用环境 (不带编号)
\newenvironment{exampleexercise}[1][]{%
    \par\noindent{\songti\color{main}\textbf{#1}}\ \rmfamily %
}{%
    \par\ignorespacesafterend
}


% --- End of config.tex ---

% ===================================================================
%  正文开始
% ===================================================================
\begin{document}

% --- 生成标题区 ---
\makeHomeworkTitle

% --- 说明框 ---
% \begin{tcolorbox}[
%     hintstyle,
%     colframe=winered, 
%     colback=winered!5!white, 
%     colbacktitle=winered, 
%     fontupper=\bfseries, 
%     title={\textbf{重要说明}}
% ]
%     以下所有题目均为模板占位符,请自己输入实际的题目,并作答
% \end{tcolorbox}

% ===================================================================
%  作业题目区
% ===================================================================
\section{第1题}

\begin{exercise}
    设向量组$\alpha _{1} ,\alpha _{2} ,\alpha _{3}$线性无关,证明向量组$\alpha _{1} +\alpha _{2},\alpha _{2} +\alpha _{3} ,\alpha _{3} +\alpha _{1}$也线性无关.\\
      
\end{exercise}~\\

\noindent\textbf{证明:}\\


\section{第2题}
\begin{exercise}
已知向量$\alpha _{1} = (2,5,1,3),\alpha _{2} =(10,1,5,10),\alpha _{3} =(4,1,-1,1)$,向量$\alpha $满足等式\[
3(\alpha_1 - \alpha) + 2(\alpha_2 + \alpha) = 5(\alpha_3 + \alpha),
\]
求$\alpha$ .
\end{exercise}~\\
\noindent\textbf{解:}~\\

\section{第3题}
\begin{exercise}
设 $\alpha_1,\alpha_2,\ldots,\alpha_m,\beta_1,\ldots,\beta_m$ 是同维向量.若 $k_1\alpha_1 + k_2\alpha_2 + \cdots + k_m\alpha_m + k_1\beta_1 + k_2\beta_2 + \cdots + k_m\beta_m = 0$ 成立, 必有 $k_1 = k_2 = \cdots = k_m = 0.$ 问 $\alpha_1,\alpha_2,\ldots,\alpha_m,\beta_1,\ldots,\beta_m$ 是否线性无关?\\
\end{exercise}~\\
\noindent\textbf{解:}\\

\section{第5题}
\begin{exercise}
举出一个线性相关的向量组的例子,使其中存在非零向量不能用其余向量线性表出.
\end{exercise}~\\
\noindent\textbf{解:}\\

\section{第6题}
\begin{exercise}
设向量 $\beta$ 能用向量 $\alpha_1, \alpha_2, \ldots, \alpha_m$ 线性表示出,且表示式是唯一的,试用反证法证明向量组 $\alpha_1, \alpha_2, \ldots, \alpha_m$ 必线性无关.\\

\end{exercise}~\\
\noindent\textbf{证明:}\\

 

\section{第7题}~\\

\begin{exercise}
设 $A = (a_{ij})_{m \times n}, B = (b_{ij})_{n \times p},$ 且 $AB = 0,$ 试证若 $A$ 的 $n$ 个列向量线性无关,则 $B = 0;$ 若 $B$ 的 $n$ 个行向量线性无关,则 $A = 0.$\\

\end{exercise}~\\

\noindent\textbf{证明:}


\section{第8题}
\begin{exercise}
用初等行变换将下列矩阵化成阶梯形矩阵,并求它们的秩:
\[
(3)\begin{pmatrix}
1 & 0 & 0 & 1 & 4 \\
0 & 1 & 0 & 2 & 5 \\
0 & 0 & 1 & 3 & 6 \\
1 & 2 & 3 & 14 & 32 \\
4 & 5 & 6 & 32 & 77
\end{pmatrix} \quad 
\]

\[
(4)\begin{pmatrix}
2 & 0 & 3 & 1 & 4 \\
3 & -5 & 4 & 2 & 7 \\
1 & 5 & 2 & 0 & 1
\end{pmatrix} \quad 
\]
\end{exercise}~\\
\noindent\textbf{解:}\\


\section{第9题}
\begin{exercise}
判断下列向量组是否线性相关,并求出它们的秩:\\
(2) $\alpha_1 = (1, -2, 3, -4), \alpha_2 = (0, 1, -1, 1), \\
\alpha_3 = (1, 3, 0, -1), \alpha_4 = (0, -7, 3, 1);$

\end{exercise}~\\
\noindent\textbf{解:}\\


\section{第16题}
\begin{exercise}
选择 $\lambda$ 的值使方程组有解,并解之。
\[
(1)\begin{cases}
2x_1 - x_2 + x_3 + x_4 = 1, \\
x_1 + 2x_2 - x_3 + 4x_4 = 2, \\
x_1 + 7x_2 - 4x_3 + 11x_4 = \lambda; \\
\end{cases}
\]

\end{exercise}~\\
\noindent\textbf{解:}

\section{第17题}
\begin{exercise}
设线性方程组
\[
\begin{cases}
a_{11}x_1 + a_{12}x_2 + \cdots + a_{1n}x_n = b_1, \\
a_{21}x_1 + a_{22}x_2 + \cdots + a_{2n}x_n = b_2, \\
\vdots \\
a_{n1}x_1 + a_{n2}x_2 + \cdots + a_{nn}x_n = b_n,
\end{cases}
\]

系统矩阵的秩与矩阵
\[
\begin{pmatrix}
a_{11} & a_{12} & \cdots & a_{1n} & b_1 \\
a_{21} & a_{22} & \cdots & a_{2n} & b_2 \\
\vdots & \vdots & \quad & \vdots & \vdots \\
a_{n1} & a_{n2} & \cdots & a_{nn} & b_n \\
b_1 & b_2 & \cdots & b_n & 0
\end{pmatrix}
\]
的秩相等.证明这个方程组有解.
\end{exercise}~\\
\noindent\textbf{证明:}

\section{第21题}
\begin{exercise}
求下列齐次线性方程组的一个基础解系及通解:

(1)
\[
\left\{
\begin{matrix}
3x_{1}+2x_{2}-5x_{3}+4x_{4}=0,\\
3x_{1}-x_{2}+3x_{3}-3x_{4}=0,\\
3x_{1}+5x_{2}-13x_{3}+11x_{4}=0;
\end{matrix}
\right.
\]

(2)
\[
\left\{
\begin{matrix}
2x_{1}-4x_{2}+5x_{3}+3x_{4}=0,\\
3x_{1}-6x_{2}+4x_{3}+2x_{4}=0,\\
4x_{1}-8x_{2}+17x_{3}+11x_{4}=0;
\end{matrix}
\right.
\]

\end{exercise}~\\
\noindent\textbf{解:}
\section{第27题}
\begin{exercise}
设 $A$ 为 $n$ 阶矩阵,且 $A^2 = A$,证明
\[
r_{A} +r_{(A-E)}  = n.
\]
\end{exercise}~\\
\noindent\textbf{证明:}
\end{document}
% ===================================================================
%  文档结束
% ===================================================================