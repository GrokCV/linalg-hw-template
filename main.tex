% !TEX program = xelatex
% ===================================================================
% 《线性代数》课程作业模板
% ===================================================================
%
%  【学生使用说明】
%  1. 在下方的【学生信息区】将 "请在此处填写..." 替换为你的真实信息。
%  2. 注意:以下所有题目均为模板占位符,请自己输入实际的题目,并作答.
%
% ===================================================================

\documentclass[a4paper, 12pt]{ctexart}

% --- 课程信息 (通常无需修改) ---
\newcommand{\CourseInfo}{线性代数} % 用于标题行的大标题
\newcommand{\CourseName}{线性代数·2025秋·第 X 章}               % 用于页眉左边
\newcommand{\InstructorName}{戴一冕}
\newcommand{\AssignmentTitle}{第 X 章 \quad XXX}

% ===================================================================
%  第一步:请在这里填写你的学生信息
% ===================================================================
\newcommand{\StudentName}{请在此处填写你的姓名}
\newcommand{\StudentID}{请在此处填写你的学号}
\newcommand{\StudentClass}{请在此处填写你的班级}
\newcommand{\AssignmentDate}{2025年10月13日} % 默认是提交日期

% --- 导入所有配置 (请勿修改) ---
% ===================================================================
%  config.tex -- The All-in-One Style & Info Configuration File
% ===================================================================

% --- PART 1: CORE PACKAGES ---
\usepackage[left=2.5cm, right=2.5cm, top=2.5cm, bottom=3cm]{geometry}
\usepackage{fancyhdr}
\usepackage{lastpage}
\usepackage{hyperref}
\usepackage[table]{xcolor}
\usepackage{amsmath, amssymb}
\usepackage{tikz}
\usetikzlibrary{shadows}
\usepackage[shortlabels]{enumitem}
\usepackage[many]{tcolorbox}
\usepackage{pifont, bbding}


% --- PART 2: USER INFORMATION AREA (WITH DEFAULTS) ---
\providecommand{\CourseInfo}{(请在main.tex中填写课程信息)}
\providecommand{\CourseName}{(课程名)}
\providecommand{\InstructorName}{(教师名)}
\providecommand{\AssignmentTitle}{(作业标题)}
\providecommand{\StudentName}{(请在main.tex中填写姓名)}
\providecommand{\StudentID}{(请在main.tex中填写学号)}
\providecommand{\StudentClass}{(请在main.tex中填写班级)}
\providecommand{\AssignmentDate}{\today}




% --- PART 3: HEADER, FOOTER, AND HYPERLINK SETUP ---
\setlength{\headheight}{15pt}
\pagestyle{fancy}
\fancyhf{}
\fancyhead[L]{\CourseName}
\fancyhead[R]{主讲: \InstructorName}
\fancyfoot[L]{课程主页:\href{https://grokcv.ai/teaching/}{https://grokcv.ai/teaching/}}
\fancyfoot[C]{}
\fancyfoot[R]{第 \thepage\ 页 / 共 \pageref{LastPage} 页}
\renewcommand{\headrulewidth}{0.4pt}
\renewcommand{\footrulewidth}{0.4pt}
\hypersetup{colorlinks, linkcolor=winered, citecolor=winered, urlcolor=winered}


% --- PART 4: AUTOMATIC TITLE BLOCK COMMAND ---
\newcommand{\makeHomeworkTitle}{%
    \vspace*{0.2cm}
    \begin{center}
        {\huge \songti \bfseries \CourseInfo} \\ \vspace{1.5em} 
        {\Large \songti \bfseries \AssignmentTitle} \\ \vspace{2.5em}
    \end{center}
    \noindent
    \begin{tabular*}{\textwidth}{@{\extracolsep{\fill}} ll}
        \songti 姓名: \StudentName & \songti 班级: \StudentClass \\ 
        \songti 学号: \StudentID & \songti 日期: \AssignmentDate \\ 
    \end{tabular*}
    \vspace{2em}\hrule\vspace{2em}
}


% --- PART 5: VISUAL STYLE SETUP (Colors, Theorems, Lists) ---
% 核心颜色定义
\definecolor{structurecolor}{RGB}{51,102,153}
\definecolor{main}{RGB}{0,102,51}
\definecolor{second}{RGB}{204,102,0}
\definecolor{third}{RGB}{0,128,128}
\definecolor{winered}{rgb}{0.5,0,0}

% 定理、定义环境
\renewcommand{\proofname}{证明}
\newcommand{\definitionname}{定义}
\newcommand{\theoremname}{定理}
\newcommand{\exercisename}{练习}
\newcommand{\hintname}{提示 \ding{182}}

% 1统一使用宋体 (fontupper={\songti ...})
\tcbset{
    common/.style={fontupper={\songti\fontseries{m}\selectfont}, coltitle=white, colback=gray!5, boxrule=0.5pt, fonttitle=\songti, enhanced, breakable, top=8pt, before skip=10pt, after skip=10pt, attach boxed title to top left={yshift=-0.1in, xshift=0.15in}, boxed title style={boxrule=0pt, colframe=white, arc=0pt, outer arc=0pt}, separator sign={.}},
    defstyle/.style={common, colframe=main, colback=main!5, colbacktitle=main},
    thmstyle/.style={common, colframe=second, colback=second!5, colbacktitle=second},
    hintstyle/.style={common, colframe=third, colback=third!5, colbacktitle=third}
}
\DeclareTColorBox[auto counter, number within=section]{definition}{ o m }{defstyle, IfValueTF={#1}{title={\definitionname~\thetcbcounter\ (#1)}}{title={\definitionname~\thetcbcounter}}, IfValueT={#2}{label={#2}}}
\DeclareTColorBox[auto counter, number within=section]{theorem}{ o m }{thmstyle, IfValueTF={#1}{title={\theoremname~\thetcbcounter\ (#1)}}{title={\theoremname~\thetcbcounter}}, IfValueT={#2}{label={#2}}}

\newenvironment{hint}[1][]{%
    \def\temphintarg{#1}%
    \def\tempempty{}%
    \ifx\temphintarg\tempempty
        \begin{tcolorbox}[hintstyle, title={\hintname}]
    \else
        \begin{tcolorbox}[hintstyle, title={\hintname~(#1)}]
    \fi
}{%
    \end{tcolorbox}
}


% 练习、证明环境
\newcounter{exercise}[section]
\renewcommand{\theexercise}{\thesection.\arabic{exercise}}
\newenvironment{exercise}[1][]{%
    \par\refstepcounter{exercise}%
    \noindent{\songti\color{main}\exercisename\ \theexercise #1}\ \rmfamily %
}{%
    \par\ignorespacesafterend
}

% 2. [新增] 作答范例专用环境 (不带编号)
\newenvironment{exampleexercise}[1][]{%
    \par\noindent{\songti\color{main}\textbf{#1}}\ \rmfamily %
}{%
    \par\ignorespacesafterend
}


% --- End of config.tex ---

% ===================================================================
%  正文开始
% ===================================================================
\begin{document}

% --- 生成标题区 ---
\makeHomeworkTitle

% --- 说明框 ---
\begin{tcolorbox}[
    hintstyle,
    colframe=winered, 
    colback=winered!5!white, 
    colbacktitle=winered, 
    fontupper=\bfseries, 
    title={\textbf{重要说明}}
]
    以下所有题目均为模板占位符,请自己输入实际的题目,并作答
\end{tcolorbox}

% ===================================================================
%  作业题目区
% ===================================================================

\section{选择题}

\begin{exercise}
    设 $A$ 为 3 阶方阵,且 $\det(A) = 2$。则 $\det(2A^{-1})$ 的值是多少?
    \begin{enumerate}[A.]
        \item 2
        \item 4
        \item 8
        \item 1/4
    \end{enumerate}
\end{exercise}
\noindent\textbf{答案:} (       )

\vspace{0.5cm}

\begin{exercise}
    向量组 $\alpha_1, \alpha_2, \dots, \alpha_s$ 线性无关的充要条件是:
    \begin{enumerate}[A.]
        \item $\alpha_1, \alpha_2, \dots, \alpha_s$ 中任意两个向量都线性无关。
        \item $\alpha_1, \alpha_2, \dots, \alpha_s$ 中不含零向量。
        \item $\alpha_1, \alpha_2, \dots, \alpha_s$ 中任意一个向量都不能由其余 $s-1$ 个向量线性表示。
        \item $\alpha_1, \alpha_2, \dots, \alpha_s$ 中存在一个向量不能由其余 $s-1$ 个向量线性表示。
    \end{enumerate}
\end{exercise}
\noindent\textbf{答案:} (       )

\vspace{1cm}

\section{填空题}

\begin{exercise}
    已知 3 阶矩阵 $A = \begin{pmatrix} 1 & 0 & 1 \\ 0 & 2 & 0 \\ 1 & 0 & 1 \end{pmatrix}$,则矩阵 $A$ 的秩 $r(A) = $ \underline{\hspace{2cm}}.
\end{exercise}

\vspace{0.5cm}

\begin{exercise}
    若 $n$ 元齐次线性方程组 $A_{m \times n}x=0$ 满足 $r(A)=n$,则该方程组的解为 \underline{\hspace{3cm}}.
\end{exercise}

\vspace{0.5cm}

\begin{exercise}
    设 $A$ 为 $n$ 阶方阵,若 $\det(A) = d \neq 0$,则其伴随矩阵 $A^*$ 的行列式 $\det(A^*) = $ \underline{\hspace{2cm}}.
\end{exercise}

\vspace{1cm}

\section{计算与证明题}

\begin{exampleexercise}[作答范例]
    计算下列三阶行列式的值:
    $$
    D = \begin{vmatrix}
    a & b & c \\
    c & a & b \\
    b & c & a
    \end{vmatrix}
    $$
\end{exampleexercise}

\begin{hint}
    观察行列式的结构,尝试将所有列加到某一列上,看看能否创造出公因子。
\end{hint}

\noindent\textbf{解:}
本题的核心思路是利用行列式的性质进行化简。观察到每一行的元素之和均为 $a+b+c$。
因此,我们将第 2 列和第 3 列加到第 1 列上(记为 $c_1 + c_2 + c_3$)。
\begin{align*}
    D &= \begin{vmatrix}
        a & b & c \\
        c & a & b \\
        b & c & a
        \end{vmatrix} \\
        &\xrightarrow{c_1+c_2+c_3} \begin{vmatrix}
        a+b+c & b & c \\
        c+a+b & a & b \\
        b+c+a & c & a
        \end{vmatrix} && \text{(性质 5:列的倍加,值不变)} \\
        &= (a+b+c) \begin{vmatrix}
        1 & b & c \\
        1 & a & b \\
        1 & c & a
        \end{vmatrix} && \text{(性质 4:提取第 1 列的公因子)} \\
        &\xrightarrow[r_3-r_1]{r_2-r_1} (a+b+c) \begin{vmatrix}
        1 & b & c \\
        0 & a-b & b-c \\
        0 & c-b & a-c
        \end{vmatrix} && \text{(性质 5:行的倍加,值不变)} \\
        &= (a+b+c) [ (a-b)(a-c) - (b-c)(c-b) ] && \text{(按第 1 列展开)} \\
        &= (a+b+c) (a^2+b^2+c^2-ab-bc-ca)
\end{align*}
最终结果亦可写作 $a^3+b^3+c^3-3abc$。

\vspace{1cm}

\begin{exercise}
    已知矩阵 $A = \begin{pmatrix} 2 & 1 \\ 1 & 1 \end{pmatrix}$, $B = \begin{pmatrix} 3 & 2 \\ 1 & 0 \end{pmatrix}$,求解矩阵方程 $AX=B$。
\end{exercise}

\begin{hint}
    当系数矩阵 $A$ 可逆时,方程 $AX=B$ 的解为 $X=A^{-1}B$。
\end{hint}

\noindent\textbf{解:}
% 在这里写下你的解题步骤...

\vspace{1cm}


\begin{exercise}
    已知矩阵 $A = \begin{pmatrix} 2 & -1 & 2 \\ 5 & -3 & 3 \\ -1 & 0 & -2 \end{pmatrix}$。
    \begin{enumerate}
        \item 求矩阵 $A$ 的所有特征值。
        \item 求每个特征值对应的全部特征向量。
    \end{enumerate}
\end{exercise}

\begin{hint}
    特征向量是特征方程 $(\lambda I - A)x = 0$ 的非零解。你需要对每个求出的特征值 $\lambda_i$,求解对应的齐次线性方程组。
\end{hint}

\noindent\textbf{解:}
% 在这里写下你的解题步骤...


\end{document}
% ===================================================================
%  文档结束
% ===================================================================