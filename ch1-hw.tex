% !TEX program = xelatex
% ===================================================================
% 《线性代数》课程作业模板
% ===================================================================
%
%  【学生使用说明】
%  1. 在下方的【学生信息区】将 "请在此处填写..." 替换为你的真实信息。
%  2. 注意:以下所有题目均为模板占位符,请自己输入实际的题目,并作答.
%
% ===================================================================


\documentclass[a4paper, 12pt]{ctexart}

\usepackage{mathdots}
\usepackage{mathtools} % 加载 mathtools 会自动加载 amsmath

% --- 课程信息 (通常无需修改) ---
\newcommand{\CourseInfo}{线性代数} % 用于标题行的大标题
\newcommand{\CourseName}{线性代数·2025秋·第 X 章}               % 用于页眉左边
\newcommand{\InstructorName}{戴一冕}
\newcommand{\AssignmentTitle}{第 1 章 \quad xxx}

% ===================================================================
%  第一步:请在这里填写你的学生信息
% ===================================================================
\newcommand{\StudentName}{xxx}
\newcommand{\StudentID}{xxxxxx}
\newcommand{\StudentClass}{计网模拟1-3班}
\newcommand{\AssignmentDate}{2025年10月13日} % 默认是提交日期

% --- 导入所有配置 (请勿修改) ---
% ===================================================================
%  config.tex -- The All-in-One Style & Info Configuration File
% ===================================================================

% --- PART 1: CORE PACKAGES ---
\usepackage[left=2.5cm, right=2.5cm, top=2.5cm, bottom=3cm]{geometry}
\usepackage{fancyhdr}
\usepackage{lastpage}
\usepackage{hyperref}
\usepackage[table]{xcolor}
\usepackage{amsmath, amssymb}
\usepackage{tikz}
\usetikzlibrary{shadows}
\usepackage[shortlabels]{enumitem}
\usepackage[many]{tcolorbox}
\usepackage{pifont, bbding}


% --- PART 2: USER INFORMATION AREA (WITH DEFAULTS) ---
\providecommand{\CourseInfo}{(请在main.tex中填写课程信息)}
\providecommand{\CourseName}{(课程名)}
\providecommand{\InstructorName}{(教师名)}
\providecommand{\AssignmentTitle}{(作业标题)}
\providecommand{\StudentName}{(请在main.tex中填写姓名)}
\providecommand{\StudentID}{(请在main.tex中填写学号)}
\providecommand{\StudentClass}{(请在main.tex中填写班级)}
\providecommand{\AssignmentDate}{\today}




% --- PART 3: HEADER, FOOTER, AND HYPERLINK SETUP ---
\setlength{\headheight}{15pt}
\pagestyle{fancy}
\fancyhf{}
\fancyhead[L]{\CourseName}
\fancyhead[R]{主讲: \InstructorName}
\fancyfoot[L]{课程主页:\href{https://grokcv.ai/teaching/}{https://grokcv.ai/teaching/}}
\fancyfoot[C]{}
\fancyfoot[R]{第 \thepage\ 页 / 共 \pageref{LastPage} 页}
\renewcommand{\headrulewidth}{0.4pt}
\renewcommand{\footrulewidth}{0.4pt}
\hypersetup{colorlinks, linkcolor=winered, citecolor=winered, urlcolor=winered}


% --- PART 4: AUTOMATIC TITLE BLOCK COMMAND ---
\newcommand{\makeHomeworkTitle}{%
    \vspace*{0.2cm}
    \begin{center}
        {\huge \songti \bfseries \CourseInfo} \\ \vspace{1.5em} 
        {\Large \songti \bfseries \AssignmentTitle} \\ \vspace{2.5em}
    \end{center}
    \noindent
    \begin{tabular*}{\textwidth}{@{\extracolsep{\fill}} ll}
        \songti 姓名: \StudentName & \songti 班级: \StudentClass \\ 
        \songti 学号: \StudentID & \songti 日期: \AssignmentDate \\ 
    \end{tabular*}
    \vspace{2em}\hrule\vspace{2em}
}


% --- PART 5: VISUAL STYLE SETUP (Colors, Theorems, Lists) ---
% 核心颜色定义
\definecolor{structurecolor}{RGB}{51,102,153}
\definecolor{main}{RGB}{0,102,51}
\definecolor{second}{RGB}{204,102,0}
\definecolor{third}{RGB}{0,128,128}
\definecolor{winered}{rgb}{0.5,0,0}

% 定理、定义环境
\renewcommand{\proofname}{证明}
\newcommand{\definitionname}{定义}
\newcommand{\theoremname}{定理}
\newcommand{\exercisename}{练习}
\newcommand{\hintname}{提示 \ding{182}}

% 1统一使用宋体 (fontupper={\songti ...})
\tcbset{
    common/.style={fontupper={\songti\fontseries{m}\selectfont}, coltitle=white, colback=gray!5, boxrule=0.5pt, fonttitle=\songti, enhanced, breakable, top=8pt, before skip=10pt, after skip=10pt, attach boxed title to top left={yshift=-0.1in, xshift=0.15in}, boxed title style={boxrule=0pt, colframe=white, arc=0pt, outer arc=0pt}, separator sign={.}},
    defstyle/.style={common, colframe=main, colback=main!5, colbacktitle=main},
    thmstyle/.style={common, colframe=second, colback=second!5, colbacktitle=second},
    hintstyle/.style={common, colframe=third, colback=third!5, colbacktitle=third}
}
\DeclareTColorBox[auto counter, number within=section]{definition}{ o m }{defstyle, IfValueTF={#1}{title={\definitionname~\thetcbcounter\ (#1)}}{title={\definitionname~\thetcbcounter}}, IfValueT={#2}{label={#2}}}
\DeclareTColorBox[auto counter, number within=section]{theorem}{ o m }{thmstyle, IfValueTF={#1}{title={\theoremname~\thetcbcounter\ (#1)}}{title={\theoremname~\thetcbcounter}}, IfValueT={#2}{label={#2}}}

\newenvironment{hint}[1][]{%
    \def\temphintarg{#1}%
    \def\tempempty{}%
    \ifx\temphintarg\tempempty
        \begin{tcolorbox}[hintstyle, title={\hintname}]
    \else
        \begin{tcolorbox}[hintstyle, title={\hintname~(#1)}]
    \fi
}{%
    \end{tcolorbox}
}


% 练习、证明环境
\newcounter{exercise}[subsection]
\renewcommand{\theexercise}{\thesubsection.\arabic{exercise}}
\newenvironment{proof}{\par\noindent\textbf{\color{second}\proofname:}\ \itshape}{\par}
\newenvironment{exercise}[1][]{%
    \par\refstepcounter{exercise}%
    \noindent{\songti\color{main}\exercisename\ \theexercise #1}\ \rmfamily %
}{%
    \par\ignorespacesafterend
}

% 2. [新增] 作答范例专用环境 (不带编号)
\newenvironment{exampleexercise}[1][]{%
    \par\noindent{\songti\color{main}\textbf{#1}}\ \rmfamily %
}{%
    \par\ignorespacesafterend
}


% --- End of config.tex ---

% ===================================================================
%  正文开始
% ===================================================================
\begin{document}

% --- 生成标题区 ---
\makeHomeworkTitle

% --- 说明框 ---
% \begin{tcolorbox}[
%     hintstyle,
%     colframe=winered, 
%     colback=winered!5!white, 
%     colbacktitle=winered, 
%     fontupper=\bfseries, 
%     title={\textbf{重要说明}}
% ]
%     以下所有题目均为模板占位符,请自己输入实际的题目,并作答
% \end{tcolorbox}

% ===================================================================
%  作业题目区
% ===================================================================
\section{第1题}

\begin{exercise}
    计算下列各排类的逆序数,从而判定是奇排列还是偶排列:\\
    
    $(2) \quad 631254$ \\
    
     $(4) \quad 135 \cdots (2n-1)246 \cdots (2n)$
\end{exercise}~\\

\noindent\textbf{解:}


\section{第5题}
\begin{exercise}
写出四阶行列式中含因子$a_{23}$且带负号的项

\end{exercise}~\\
\noindent\textbf{解:}

\section{第6题}
\begin{exercise}
利用行列式的定义计算\\


$
(2) \quad 
\begin{bmatrix}
 a_{11}& a_{12} &a_{13}  & a_{14} &a_{15} \\
 a_{21}& a_{21} & a_{23} & a_{24} & a_{25}\\
 a_{31}& a_{31} & 0 & 0 & 0\\
 a_{41}& a_{41} & 0 & 0 & 0\\
 a_{51}& a_{51} & 0 & 0 & 0
\end{bmatrix}$~\\~\\


$(4) \quad \begin{bmatrix}
  x&y  &0  &0  &0 \\
  0&x  &y  &0  &0 \\
  0&0  &x  &y  &0 \\
  0&0  &0  &x  &y \\
  y&0  &0  &0  &x
\end{bmatrix}$

\end{exercise}~\\
\noindent\textbf{解:}



\section{第7题}
\begin{exercise}
若 $n$ 阶行列式 $\det(a_{ij})$ 为零的元多于 $n^{2}-n$ 个,则 $\det(a_{ij})=0$
\end{exercise}~\\
\noindent\textbf{解:}

\section{第8题}
\begin{exercise}
利用行列式的性质计算\\

$(2) \quad \begin{vmatrix}
  3&  1& 1& 1\\
  1&  3& 1& 1\\
  1&  1& 3& 1\\
  1&  1& 1& 3
\end{vmatrix}$\\~\\

$(3) \quad \begin{vmatrix}
  a&  b&  c& 1\\
  b&  c&  a& 1\\
  c&  a&  b& 1\\
  \frac{b+c}{2}&  \frac{c+a}{2}&  \frac{a+b}{2}& 1
\end{vmatrix}$
\end{exercise}~\\
\noindent\textbf{解:}

\section{第10题}
\begin{exercise}
不展开行列式,证明下列等式成立:\\

$\begin{bmatrix}
 \sin^{2}\alpha& \cos^{2}\alpha  &\cos2\alpha  \\
 \sin^{2}\beta & \cos^{2}\beta   &\cos2\beta  \\
 \sin^{2}\gamma& \cos^{2}\gamma  &\cos2\gamma 
\end{bmatrix}$=0
\end{exercise}~\\

\noindent\textbf{解:}


\section{第11题}
\begin{exercise}
计算下列行列式的值:\\

$(1) \quad \begin{bmatrix*}[l]
  a &  b    &  c & d \\
  a &  a+b  &  a+b+c & a+b+c+d \\
  a &  2a+b &  3a+2b+c & 4a+3b+2c+d \\
  a &  3a+b &  6a+3b+c & 10a+6b+3c+d
\end{bmatrix*}$\\~\\

$(4) \quad \begin{bmatrix}
  a_{1}-b_{1} & a_{1}-b_{2}& ...   & a_{1}-b_{n}\\
  a_{2}-b_{1}& a_{2}-b_{2} & ...&a_{2}-b_{n} \\
  a_{3}-b_{1}& a_{3}-b_{2} & ... & a_{3}-b_{n}\\
  ...& ... &  & ...\\
  a_{n}-b_{1}& a_{n}-b_{2} & ... &a_{n}-b_{n}
\end{bmatrix}$\\~\\

$(6) \quad \begin{bmatrix}
  x_{1}-m & x_{2} & ... & x_{n} \\
 x_{1} &  x_{2}-m&  ...& x_{n}  \\
  \vdots & \vdots  &  & \vdots  \\
 x_{1} &  x_{2}&  ...&x_{n}-m
\end{bmatrix}$
\end{exercise}~\\
\noindent\textbf{解:}


\section{第13题}
\begin{exercise}
计算下列 $n$ 阶行列式的值:\\


$(2) \quad \begin{vmatrix}
 1 & 2 & 3 & \cdots & n-1 & n\\
 1 & -1 & 0 & \cdots & 0 & 0\\
 0 & 2 & -2 & \cdots & 0 & 0\\
 \vdots & \vdots & \vdots &   &\vdots &\vdots \\
 0 & 0 & 0 & \cdots &n-1 &1-n
\end{vmatrix}$
\end{exercise}~\\
\noindent\textbf{解:}


\section{第14题}
\begin{exercise}
证明下列等式:\\

$(3) \quad \begin{vmatrix}
 a+x_{1}  &a  &a  & \cdots &a  &a \\
 a & a+x_{2}  & a & \cdots  & a & a\\
\vdots & \vdots & \vdots  &  & \vdots & \vdots\\
 a & a & a & \cdots & a+x_{n}  & a \\
  a& a & a & \cdots & a &a\\
  \end{vmatrix}= ax_{1}x _{2} \cdots x_{n} ;\\
$\\

  

$(4) \quad \begin{vmatrix}
a_{0}  & -1 & 0 &  \cdots & 0 & 0\\
 a_{1}  & x & -1 & \cdots &  0& 0\\
\vdots & \vdots & \vdots &  & \vdots & \vdots\\
a_{n-2}  & 0 & 0 & \cdots & x & -1\\
 a_{n-1}  & 0 & 0 & \cdots & 0 & x\\
\end{vmatrix}= a_{0} x^{n-1} +a_{1} x^{n-2} + \cdots +a_{n-1} .
$
\end{exercise}~\\
\noindent\textbf{解:}

\section{第15题}
\begin{exercise}
利用拉普拉斯定理计算行列式的值:


% $\left| A \right| = \begin{vmatrix}
% a &   &  &  &  &  &  & b\\
%   & a &  &  &  &  & b & \\
%   &   & \ddots &  &  &  \iddots&  & \\
%   &   &  & a & b &  &  & \\
%   &   &  &  b& a &  & &\\
%   &   & \iddots &  &  & \ddots &  &\\
%   & b &  &  &  &  & a & \\
% b &   &  &  &  &  &  &a
% \end{vmatrix}$


$
(4) \quad \left| A \right| =
\left|
  \begin{array}{cccccccc}
    a &   &         &   &   &         &   & b \\
      & a &         &   &   &         & b &   \\
      &   & \ddots  &   &   & \iddots &   &   \\
      &   &         & a & b &         &   &   \\
      &   &         & b & a &         &   &   \\
      &   & \iddots &   &   &  \ddots &   &   \\
      & b &         &   &   &         & a &   \\
    b &   &         &   &   &         &   & a \\
  \end{array}
\right|
\begin{array}{c}
    \left. \vphantom{
        \begin{array}{c} a \\ a \\ \vdots \\ a \end{array}
    } \right\} \text{$n$ 行} \\
    \left. \vphantom{
        \begin{array}{c} a \\ a \\ \vdots \\ a \end{array}
    } \right\} \text{$n$ 行}
\end{array}
$ (空白处的元均为 0).

% \begin{vmatrix}
% A
% \end{vmatrix}=\begin{vmatrix}
%   a&  &  &  &  &  &  & b\\
%   & a &  &  &  &  & b & \\
%   &  & ... &  &  &  ...&  & \\
%   &  &  & a & b &  &  & \\
%   &  &  &  b& a &  & &\\
%   &  & ... &  &  & ... &  &\\
%   & b &  &  &  &  & a & \\
%   b&  &  &  &  &  &  &a
% \end{vmatrix}(2n行)(空白处的元均为0)
\end{exercise}~\\
\noindent\textbf{解:}


\section{第16题}
\begin{exercise}
设 $P(x) = \begin{bmatrix}
 1 &  x& x^{2}  & \cdots  &x^{n-1} \\
  1& a_{1} &  a_{1} ^{2}& \cdots &a_{1} ^{n-1}\\
 1 & a_{2} &  a_{2}^{2}& \cdots &a_{2}^{n-1} \\
 \cdots & \cdots &  \cdots &  & \cdots\\
 1 & a_{n-1} & a_{n-1}^{2} & \cdots &a_{n-1}^{n-1}
\end{bmatrix}$, 其中 $a_{1}, a_{2}, \cdots, a_{n-1}$ 为互不相同之实数.\\
(1) 证明 $P(x)$ 是 $n-1$ 次多项式;\\
(2) 求 $P(x)$ 的根.
\end{exercise}~\\
\noindent\textbf{解:}










\end{document}
% ===================================================================
%  文档结束
% ===================================================================