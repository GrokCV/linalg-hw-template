% ===================================================================
%  config.tex -- The All-in-One Style & Info Configuration File
% ===================================================================

% --- PART 1: CORE PACKAGES ---
\usepackage[left=2.5cm, right=2.5cm, top=2.5cm, bottom=3cm]{geometry}
\usepackage{fancyhdr}
\usepackage{lastpage}
\usepackage{hyperref}
\usepackage[table]{xcolor}
\usepackage{amsmath, amssymb}
\usepackage{tikz}
\usetikzlibrary{shadows}
\usepackage[shortlabels]{enumitem}
\usepackage[many]{tcolorbox}
\usepackage{pifont, bbding}


% --- PART 2: USER INFORMATION AREA (WITH DEFAULTS) ---
\providecommand{\CourseInfo}{(请在main.tex中填写课程信息)}
\providecommand{\CourseName}{(课程名)}
\providecommand{\InstructorName}{(教师名)}
\providecommand{\AssignmentTitle}{(作业标题)}
\providecommand{\StudentName}{(请在main.tex中填写姓名)}
\providecommand{\StudentID}{(请在main.tex中填写学号)}
\providecommand{\StudentClass}{(请在main.tex中填写班级)}
\providecommand{\AssignmentDate}{\today}




% --- PART 3: HEADER, FOOTER, AND HYPERLINK SETUP ---
\setlength{\headheight}{15pt}
\pagestyle{fancy}
\fancyhf{}
\fancyhead[L]{\CourseName}
\fancyhead[R]{主讲: \InstructorName}
\fancyfoot[L]{课程主页:\href{https://grokcv.ai/teaching/}{https://grokcv.ai/teaching/}}
\fancyfoot[C]{}
\fancyfoot[R]{第 \thepage\ 页 / 共 \pageref{LastPage} 页}
\renewcommand{\headrulewidth}{0.4pt}
\renewcommand{\footrulewidth}{0.4pt}
\hypersetup{colorlinks, linkcolor=winered, citecolor=winered, urlcolor=winered}


% --- PART 4: AUTOMATIC TITLE BLOCK COMMAND ---
\newcommand{\makeHomeworkTitle}{%
    \vspace*{0.2cm}
    \begin{center}
        {\huge \songti \bfseries \CourseInfo} \\ \vspace{1.5em} 
        {\Large \songti \bfseries \AssignmentTitle} \\ \vspace{2.5em}
    \end{center}
    \noindent
    \begin{tabular*}{\textwidth}{@{\extracolsep{\fill}} ll}
        \songti 姓名: \StudentName & \songti 班级: \StudentClass \\ 
        \songti 学号: \StudentID & \songti 日期: \AssignmentDate \\ 
    \end{tabular*}
    \vspace{2em}\hrule\vspace{2em}
}


% --- PART 5: VISUAL STYLE SETUP (Colors, Theorems, Lists) ---
% 核心颜色定义
\definecolor{structurecolor}{RGB}{51,102,153}
\definecolor{main}{RGB}{0,102,51}
\definecolor{second}{RGB}{204,102,0}
\definecolor{third}{RGB}{0,128,128}
\definecolor{winered}{rgb}{0.5,0,0}

% 定理、定义环境
\renewcommand{\proofname}{证明}
\newcommand{\definitionname}{定义}
\newcommand{\theoremname}{定理}
\newcommand{\exercisename}{练习}
\newcommand{\hintname}{提示 \ding{182}}

% 1统一使用宋体 (fontupper={\songti ...})
\tcbset{
    common/.style={fontupper={\songti\fontseries{m}\selectfont}, coltitle=white, colback=gray!5, boxrule=0.5pt, fonttitle=\songti, enhanced, breakable, top=8pt, before skip=10pt, after skip=10pt, attach boxed title to top left={yshift=-0.1in, xshift=0.15in}, boxed title style={boxrule=0pt, colframe=white, arc=0pt, outer arc=0pt}, separator sign={.}},
    defstyle/.style={common, colframe=main, colback=main!5, colbacktitle=main},
    thmstyle/.style={common, colframe=second, colback=second!5, colbacktitle=second},
    hintstyle/.style={common, colframe=third, colback=third!5, colbacktitle=third}
}
\DeclareTColorBox[auto counter, number within=section]{definition}{ o m }{defstyle, IfValueTF={#1}{title={\definitionname~\thetcbcounter\ (#1)}}{title={\definitionname~\thetcbcounter}}, IfValueT={#2}{label={#2}}}
\DeclareTColorBox[auto counter, number within=section]{theorem}{ o m }{thmstyle, IfValueTF={#1}{title={\theoremname~\thetcbcounter\ (#1)}}{title={\theoremname~\thetcbcounter}}, IfValueT={#2}{label={#2}}}

\newenvironment{hint}[1][]{%
    \def\temphintarg{#1}%
    \def\tempempty{}%
    \ifx\temphintarg\tempempty
        \begin{tcolorbox}[hintstyle, title={\hintname}]
    \else
        \begin{tcolorbox}[hintstyle, title={\hintname~(#1)}]
    \fi
}{%
    \end{tcolorbox}
}


% 练习、证明环境
\newcounter{exercise}[subsection]
\renewcommand{\theexercise}{\thesubsection.\arabic{exercise}}
\newenvironment{proof}{\par\noindent\textbf{\color{second}\proofname:}\ \itshape}{\par}
\newenvironment{exercise}[1][]{%
    \par\refstepcounter{exercise}%
    \noindent{\songti\color{main}\exercisename\ \theexercise #1}\ \rmfamily %
}{%
    \par\ignorespacesafterend
}

% 2. [新增] 作答范例专用环境 (不带编号)
\newenvironment{exampleexercise}[1][]{%
    \par\noindent{\songti\color{main}\textbf{#1}}\ \rmfamily %
}{%
    \par\ignorespacesafterend
}


% --- End of config.tex ---