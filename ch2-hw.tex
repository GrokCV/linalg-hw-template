% !TEX program = xelatex
% ===================================================================
% 《线性代数》课程作业模板
% ===================================================================
%
%  【学生使用说明】
%  1. 在下方的【学生信息区】将 "请在此处填写..." 替换为你的真实信息。
%  2. 注意:以下所有题目均为模板占位符,请自己输入实际的题目,并作答.
%
% ===================================================================


\documentclass[a4paper, 12pt]{ctexart}

% --- 课程信息 (通常无需修改) ---
\newcommand{\CourseInfo}{线性代数} % 用于标题行的大标题
\newcommand{\CourseName}{线性代数·2025秋·第 2 章}               % 用于页眉左边
\newcommand{\InstructorName}{戴一冕}
\newcommand{\AssignmentTitle}{第 2 章 \quad 矩阵代数}

% ===================================================================
%  第一步:请在这里填写你的学生信息
% ===================================================================
\newcommand{\StudentName}{xxx}
\newcommand{\StudentID}{xxxxxx}
\newcommand{\AssignmentDate}{2025年xx月xx日} % 默认是提交日期

% --- 导入所有配置 (请勿修改) ---
% ===================================================================
%  config.tex -- The All-in-One Style & Info Configuration File
% ===================================================================

% --- PART 1: CORE PACKAGES ---
\usepackage[left=2.5cm, right=2.5cm, top=2.5cm, bottom=3cm]{geometry}
\usepackage{fancyhdr}
\usepackage{lastpage}
\usepackage{hyperref}
\usepackage[table]{xcolor}
\usepackage{amsmath, amssymb}
\usepackage{tikz}
\usetikzlibrary{shadows}
\usepackage[shortlabels]{enumitem}
\usepackage[many]{tcolorbox}
\usepackage{pifont, bbding}


% --- PART 2: USER INFORMATION AREA (WITH DEFAULTS) ---
\providecommand{\CourseInfo}{(请在main.tex中填写课程信息)}
\providecommand{\CourseName}{(课程名)}
\providecommand{\InstructorName}{(教师名)}
\providecommand{\AssignmentTitle}{(作业标题)}
\providecommand{\StudentName}{(请在main.tex中填写姓名)}
\providecommand{\StudentID}{(请在main.tex中填写学号)}
\providecommand{\StudentClass}{(请在main.tex中填写班级)}
\providecommand{\AssignmentDate}{\today}




% --- PART 3: HEADER, FOOTER, AND HYPERLINK SETUP ---
\setlength{\headheight}{15pt}
\pagestyle{fancy}
\fancyhf{}
\fancyhead[L]{\CourseName}
\fancyhead[R]{主讲: \InstructorName}
\fancyfoot[L]{课程主页:\href{https://grokcv.ai/teaching/}{https://grokcv.ai/teaching/}}
\fancyfoot[C]{}
\fancyfoot[R]{第 \thepage\ 页 / 共 \pageref{LastPage} 页}
\renewcommand{\headrulewidth}{0.4pt}
\renewcommand{\footrulewidth}{0.4pt}
\hypersetup{colorlinks, linkcolor=winered, citecolor=winered, urlcolor=winered}


% --- PART 4: AUTOMATIC TITLE BLOCK COMMAND ---
\newcommand{\makeHomeworkTitle}{%
    \vspace*{0.2cm}
    \begin{center}
        {\huge \songti \bfseries \CourseInfo} \\ \vspace{1.5em} 
        {\Large \songti \bfseries \AssignmentTitle} \\ \vspace{2.5em}
    \end{center}
    \noindent
    \begin{tabular*}{\textwidth}{@{\extracolsep{\fill}} ll}
        \songti 姓名: \StudentName & \songti 班级: \StudentClass \\ 
        \songti 学号: \StudentID & \songti 日期: \AssignmentDate \\ 
    \end{tabular*}
    \vspace{2em}\hrule\vspace{2em}
}


% --- PART 5: VISUAL STYLE SETUP (Colors, Theorems, Lists) ---
% 核心颜色定义
\definecolor{structurecolor}{RGB}{51,102,153}
\definecolor{main}{RGB}{0,102,51}
\definecolor{second}{RGB}{204,102,0}
\definecolor{third}{RGB}{0,128,128}
\definecolor{winered}{rgb}{0.5,0,0}

% 定理、定义环境
\renewcommand{\proofname}{证明}
\newcommand{\definitionname}{定义}
\newcommand{\theoremname}{定理}
\newcommand{\exercisename}{练习}
\newcommand{\hintname}{提示 \ding{182}}

% 1统一使用宋体 (fontupper={\songti ...})
\tcbset{
    common/.style={fontupper={\songti\fontseries{m}\selectfont}, coltitle=white, colback=gray!5, boxrule=0.5pt, fonttitle=\songti, enhanced, breakable, top=8pt, before skip=10pt, after skip=10pt, attach boxed title to top left={yshift=-0.1in, xshift=0.15in}, boxed title style={boxrule=0pt, colframe=white, arc=0pt, outer arc=0pt}, separator sign={.}},
    defstyle/.style={common, colframe=main, colback=main!5, colbacktitle=main},
    thmstyle/.style={common, colframe=second, colback=second!5, colbacktitle=second},
    hintstyle/.style={common, colframe=third, colback=third!5, colbacktitle=third}
}
\DeclareTColorBox[auto counter, number within=section]{definition}{ o m }{defstyle, IfValueTF={#1}{title={\definitionname~\thetcbcounter\ (#1)}}{title={\definitionname~\thetcbcounter}}, IfValueT={#2}{label={#2}}}
\DeclareTColorBox[auto counter, number within=section]{theorem}{ o m }{thmstyle, IfValueTF={#1}{title={\theoremname~\thetcbcounter\ (#1)}}{title={\theoremname~\thetcbcounter}}, IfValueT={#2}{label={#2}}}

\newenvironment{hint}[1][]{%
    \def\temphintarg{#1}%
    \def\tempempty{}%
    \ifx\temphintarg\tempempty
        \begin{tcolorbox}[hintstyle, title={\hintname}]
    \else
        \begin{tcolorbox}[hintstyle, title={\hintname~(#1)}]
    \fi
}{%
    \end{tcolorbox}
}


% 练习、证明环境
\newcounter{exercise}[section]
\renewcommand{\theexercise}{\thesection.\arabic{exercise}}
\newenvironment{exercise}[1][]{%
    \par\refstepcounter{exercise}%
    \noindent{\songti\color{main}\exercisename\ \theexercise #1}\ \rmfamily %
}{%
    \par\ignorespacesafterend
}

% 2. [新增] 作答范例专用环境 (不带编号)
\newenvironment{exampleexercise}[1][]{%
    \par\noindent{\songti\color{main}\textbf{#1}}\ \rmfamily %
}{%
    \par\ignorespacesafterend
}


% --- End of config.tex ---

% ===================================================================
%  正文开始
% ===================================================================
\begin{document}

\makeHomeworkTitle

% ===================================================================
%  作业题目区
% ===================================================================
\section{第2题}
\begin{exercise}[1]
    设$\alpha_{1} = \left (  1,1,1,1\right )$,$\alpha_{2} = \left (  1,-1,+1,-1\right )$ ,$\alpha_{3} = \left ( 3,1,3,1 \right )$.\\
    
    $(1)$ $\quad$ 求$2\alpha_{1}$ + $\alpha_{2}$ - $\alpha_{3}$;\\

    $(2)$ $\quad$ $\alpha_{3}$能表示成$\alpha_{1}$和$\alpha_{2}$的线性组合吗?
    
    
\end{exercise}~\\

\noindent\textbf{解:}


\section{第4题}
\begin{exercise}计算下列矩阵乘积.~\\


$(1)$ $\begin{pmatrix}
  3& -2\\
  0&  1\\
  2& 4\\
  -1& 0 
\end{pmatrix} 
\begin{pmatrix}
  2&  1& -1\\
  0&  -1& 2
\end{pmatrix} $~\\~\\

$(4)$
$\left ( x,y,1 \right ) 
\begin{pmatrix}
  a_{11}&  a_{12}& b_{1}\\
  a_{21}&  a_{22}& b_{2}\\
  b_{1}&  b_{2}& c
\end{pmatrix}
\begin{pmatrix}
  x\\
  y\\
  1
\end{pmatrix}
\left ( a_{12} = a_{21} \right ) $

\end{exercise}
\noindent\textbf{解:}

\section{第5题}
\begin{exercise}
设$A$ = $\begin{pmatrix}
 1 & 3\\
 2 & -1
\end{pmatrix}$,
$B$ = $\begin{pmatrix}
 3 & 0\\
 1 & 2
\end{pmatrix}$,
求$A^2$,$B^2$,$A^2B^2$与$(AB)^2$.
\end{exercise}
\noindent\textbf{解:}



\section{第6题}
\begin{exercise}
求$A^n$(n为正整数).~\\

$(3)$ $A$ = $\begin{pmatrix}
 a & 1 & 0 & 0\\
 0 & a & 1 & 0\\
 0 & 0 & a & 1\\
 0 & 0 & 0 & a
\end{pmatrix}$
\end{exercise}
\noindent\textbf{解:}

\section{第7题}
\begin{exercise}
已知~\\
\begin{center}
$B$ = $\begin{pmatrix}
 1 & 4 & 2\\
 0 & -3 & 2\\
 0 & 4 & 3
\end{pmatrix}$~\\
\end{center}~\\

$\hspace{1.5em}$证明$B^n$ = $\left\{\begin{matrix}
 E,\text{当$n$为偶数,}\\
 B,\text{当$n$为奇数.}
\end{matrix}\right.$

\end{exercise}
\noindent\textbf{解:}

\section{第8题}
\begin{exercise}~\\

 证明两个 $n$ 阶上三角形矩阵的乘积仍为一上三角形矩阵.
\end{exercise}

\noindent\textbf{解:}


\section{第14题}
\begin{exercise}
利用$\begin{vmatrix}
AB
\end{vmatrix} = 
\begin{vmatrix}
A
\end{vmatrix}
\begin{vmatrix}
B
\end{vmatrix}$
计算下列行列式.~\\

$(2)$ $\begin{vmatrix}
 1 & \cos(\alpha_{1}-\alpha_{2})  & \cos(\alpha_{1}-\alpha_{3}) & \cdots & \cos(\alpha_{1}-\alpha_{n})\\
 \cos(\alpha_{1}-\alpha_{2}) & 1 & \cos(\alpha_{2}-\alpha_{3}) & \cdots & \cos(\alpha_{2}-\alpha_{n})\\
 \cos(\alpha_{1}-\alpha_{3}) & \cos(\alpha_{2}-\alpha_{3}) & 1 & \cdots & \cos(\alpha_{3}-\alpha_{n})\\
 \vdots & \vdots & \vdots &  & \vdots\\
 \cos(\alpha_{1}-\alpha_{n}) & \cos(\alpha_{2}-\alpha_{n}) & \cos(\alpha_{3}-\alpha_{n}) & \cdots& 1
\end{vmatrix}$
\end{exercise}
\noindent\textbf{解:}


\section{第15题}
\begin{exercise}
求$A^{-1}$.~\\

$(1)A$ = $\begin{pmatrix}
 a & b\\
 c & d
\end{pmatrix}$\\~\\

$(3)A$ = $\begin{pmatrix}
 1 & 2 & -3\\
 0 & 1 & 2\\
 0 & 0 & 1
\end{pmatrix}$\\~\\

$(6)A$ = $\begin{pmatrix}
 2 & 1 & 0 & 0 & 0\\
 0 & 2 & 1 & 0 & 0\\
 0 & 0 & 2 & 1 & 0\\
 0 & 0 & 0 & 2 & 1\\
 0 & 0 & 0 & 0 & 2
\end{pmatrix}$
\end{exercise}
\noindent\textbf{解:}


\section{第16题}
\begin{exercise}
解下列矩阵方程,求$X$.~\\

$(2)X\begin{pmatrix}
 3 & 6\\
 4 & 8
\end{pmatrix}$ = 
$\begin{pmatrix}
 2 & 4\\
 9 & 18
\end{pmatrix}$
\end{exercise}
\noindent\textbf{解:}

\section{第17题}
\begin{exercise}
若$A$和$B$是同阶可逆矩阵,且$AB$ = $BA$,
证明:$AB^{-1}$ = $B^{-1}A$,$A^{-1}B$ = $BA^{-1}$,$A^{-1}B^{-1}$ = $B^{-1}A^{-1}$.
\end{exercise}
\noindent\textbf{解:}


\section{第20题}
\begin{exercise}
求一个二次多项式$f(x)$,使$f(1) = -1$,$f(-1) = 9$, $f(2) = 3 $.

\end{exercise}
\noindent\textbf{解:}


\section{第21题}
\begin{exercise}
用适当的方法解下列方程组.~\\

(1)$\left\{\begin{matrix}
-x+y+z+t = a \\
x-y+z+t = b \\
x+y-z+t = c \\
x+y+z-t = d
\end{matrix}\right.$
\end{exercise}
\noindent\textbf{解:}

\section{第23题}
\begin{exercise}
设$E_{ij}$,$E_{ii}$($c$)($c\ne0$),$E_{ij}$($\lambda$)($i\ne j$)为三种初等矩阵.~\\

$(1)$求它们的转置矩阵$E_{ij}^T$,$(E_{ii}(c))^T$,$(E_{ij}(\lambda))^T$.~\\

$(2)$试述$E_{ij}^TAE_{ij}$,$(E_{ii}(c))^TAE_{ii}(c)$,$(E_(ij)(\lambda))^TAE_{ij}(\lambda))$是分别对$A$施行了什么\\

样的初等变换.
\end{exercise}
\noindent\textbf{解:}

\section{第24题}
\begin{exercise}
证明可逆对称矩阵的逆矩阵也是对称矩阵.
\end{exercise}
\noindent\textbf{解:}

\section{第25题}
\begin{exercise}
若$A$是$n$阶对称矩阵,$B$是$n$阶反称矩阵,试证:~\\

$(1)A^2$,$B^2$及$AB- BA$都是对称矩阵;\\

$(2)AB$是反称矩阵的充要条件是$AB = BA$.
\end{exercise}
\noindent\textbf{解:}

\section{第29题}
\begin{exercise}
设~\\
\begin{center}
 $X$ = $\begin{pmatrix}
 0 & A\\
 C & 0
\end{pmatrix}$
\end{center}~\\

其中$A$,$C$为两个可逆矩阵,已知$A^{-1}$,$C^{-1}$,求$X^{-1}$.
\end{exercise}
\noindent\textbf{解:}

\section{第31题}
\begin{exercise}
将下列各题矩阵分块后再计算:~\\

$(2)$ $\begin{pmatrix}
 1 & 3 & 0 & 0 & 0\\
 2 & 8 & 0 & 0 & 0\\
 0 & 0 & 1 & 0 & 1\\
 0 & 0 & 2 & 3 & 2\\
 0 & 0 & 3 & 1 & 1
\end{pmatrix}^{-1}$
\end{exercise}
\noindent\textbf{解:}









\end{document}
% ===================================================================
%  文档结束
% ===================================================================